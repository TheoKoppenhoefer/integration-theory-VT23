%


\documentclass{beamer}
% alternatives: scrartcl, article or report


\usetheme{default}
\usecolortheme{default}
\usefonttheme[onlymath]{serif}



%%%%% PACKAGES

% small tweaks and nicer typography
\usepackage{microtype}
\usepackage{hyperref}

% changes language to German
% gives proper date, and correct hyphenation
%\usepackage[ngerman]{babel}
%\uselanguage{German}
%\languagepath{German}

% basic math stuff
\usepackage{mathtools}
\usepackage{amssymb}
\usepackage{amsthm}
%\usepackage{tikz-cd}
\usepackage{cancel}
\usepackage{cases}
\usepackage{dsfont}
\usepackage[extdef]{delimset} % for nice delimiters
\usepackage{centernot}


% code
%\usepackage{listings}
%\usepackage{pythonhighlight}
%\usepackage{algorithm}
%\usepackage{algpseudocode}
\usepackage{algorithm2e}
\SetAlgoSkip{medskip} % bigskip
\DontPrintSemicolon
%\RestyleAlgo{algoruled} % ruled, plain
\SetKwComment{brkcomment}{(}{)}

%\usepackage[backend=bibtex, style=ieee]{biblatex}
\usepackage[backend=biber, style=ieee]{biblatex}
%\usepackage{biblatex}
%\addbibresource{mybib.bib}

% dealing with figures
%\usepackage[figurename=Abb.]{caption}
\usepackage{subcaption}
\usepackage{wrapfig}

% display quotes correctly
\usepackage{csquotes}

% allow for any font-size, alternative mathpazo
\usepackage{mathptmx}

% enumeration
\usepackage{enumitem}

% color
\usepackage{transparent}
\usepackage{xcolor}

%%%% Graphics %%%%%

%\graphicspath{{Plots/}}

%\usepackage{booktabs}
%\usepackage{bm}
%\usepackage{minted}

% for inkscape images
%\usepackage{pdftricks}
%\begin{psinputs}
%   \usepackage{pstricks}
%   \usepackage{multido}
%\end{psinputs}
%\usepackage[pdf]{pstricks}
%\usepackage{import}



% images
\usepackage{graphicx}
\graphicspath{ {./Plots} }

% tikz
\usepackage{tikz}
\usetikzlibrary{positioning}
%\usetikzlibrary{babel}
\usepackage{pgfplots}



% Tikz librarys
%\tikzexternalise
\usetikzlibrary{external}
\tikzexternalize[prefix=cache/] % activate and define figureCache/ as cache folder
%\usetikzlibrary{patterns}
\tikzset{>=stealth}
\newcommand{\tikzmark}[3][]{\tikzset{external/export=false}\tikz[remember picture,baseline] \node [anchor=base,#1](#2) {$#3$};}
\usetikzlibrary{datavisualization}
\usetikzlibrary{datavisualization.formats.functions}



%%%%% CONFIGURATION

% prevents automatic line breaks inside of equations
% since it looks bad
\binoppenalty = \maxdimen
\relpenalty   = \maxdimen


%%%%%% PGFPLOTS %%%%%%%%%%%

%\usepgfplotslibrary{grouplots}
\usepgfplotslibrary{dateplot}


%%%%% CUSTOM COMMANDS

% real numbers via \R
% complex numbers via \C
% general field via \K
\def\C{\mathbb{C}}
\def\R{\mathbb{R}}
\def\K{\mathbb{K}}
\def\F{\mathbb{F}}
\def\Q{\mathbb{Q}}
\def\Z{\mathbb{Z}}
\def\N{\mathbb{N}}
\def\H{\mathbb{H}}
\def\e{\varepsilon}

\newcommand{\cA}{\mathcal{A}}
\newcommand{\cB}{\mathcal{B}}
\newcommand{\cC}{\mathcal{C}}
\newcommand{\cD}{\mathcal{D}}
\newcommand{\cE}{\mathcal{E}}
\newcommand{\cF}{\mathcal{F}}
\newcommand{\cG}{\mathcal{G}}
\newcommand{\cH}{\mathcal{H}}
\newcommand{\cI}{\mathcal{I}}
\newcommand{\cJ}{\mathcal{J}}
\newcommand{\cK}{\mathcal{K}}
\newcommand{\cL}{\mathcal{L}}
\newcommand{\cM}{\mathcal{M}}
\newcommand{\cN}{\mathcal{N}}
\newcommand{\cO}{\mathcal{O}}
\newcommand{\cP}{\mathcal{P}}
\newcommand{\cQ}{\mathcal{Q}}
\newcommand{\cR}{\mathcal{R}}
\newcommand{\cS}{\mathcal{S}}
\newcommand{\cT}{\mathcal{T}}
\newcommand{\cU}{\mathcal{U}}
\newcommand{\cV}{\mathcal{V}}
\newcommand{\cW}{\mathcal{W}}
\newcommand{\cX}{\mathcal{X}}
\newcommand{\cY}{\mathcal{Y}}
\newcommand{\cZ}{\mathcal{Z}}

\newcommand{\bA}{\mathbb{A}}
\newcommand{\bB}{\mathbb{B}}
\newcommand{\bC}{\mathbb{C}}
\newcommand{\bD}{\mathbb{D}}
\newcommand{\bE}{\mathbb{E}}
\newcommand{\bF}{\mathbb{F}}
\newcommand{\bG}{\mathbb{G}}
\newcommand{\bH}{\mathbb{H}}
\newcommand{\bI}{\mathbb{I}}
\newcommand{\bJ}{\mathbb{J}}
\newcommand{\bK}{\mathbb{K}}
\newcommand{\bL}{\mathbb{L}}
\newcommand{\bM}{\mathbb{M}}
\newcommand{\bN}{\mathbb{N}}
\newcommand{\bO}{\mathbb{O}}
\newcommand{\bP}{\mathbb{P}}
\newcommand{\bQ}{\mathbb{Q}}
\newcommand{\bR}{\mathbb{R}}
\newcommand{\bS}{\mathbb{S}}
\newcommand{\bT}{\mathbb{T}}
\newcommand{\bU}{\mathbb{U}}
\newcommand{\bV}{\mathbb{V}}
\newcommand{\bW}{\mathbb{W}}
\newcommand{\bX}{\mathbb{X}}
\newcommand{\bY}{\mathbb{Y}}
\newcommand{\bZ}{\mathbb{Z}}



\newcommand{\hu}{\hat{u}}
\newcommand{\hv}{\hat{v}}
\newcommand{\hV}{\hat{V}}
\newcommand{\hw}{\hat{w}}
\newcommand{\hW}{\hat{W}}
\newcommand{\hA}{\hat{A}}
\newcommand{\hC}{\hat{C}}
\newcommand{\hR}{\hat{R}}
\newcommand{\hQ}{\hat{Q}}
\newcommand{\hq}{\hat{q}}
\newcommand{\hp}{\hat{p}}
\newcommand{\hl}{\hat{\ell}}
\newcommand{\hlambda}{\hat{\lambda}}
\newcommand{\ha}{\hat{a}}
\newcommand{\hb}{\hat{b}}
\newcommand{\hs}{\hat{s}}


\newcommand{\tiS}{\tilde{S}}
\newcommand{\tiu}{\tilde{u}}
\newcommand{\tih}{\tilde{h}}
\newcommand{\tix}{\tilde{x}}
\newcommand{\tiy}{\tilde{y}}
\newcommand{\tis}{\tilde{s}}
\newcommand{\tie}{\tilde{\e}}
\newcommand{\tisigma}{\tilde{\sigma}}


\newcommand{\bartheta}{\bar{\theta}}
\newcommand{\barU}{\bar{U}}


\newcommand{\BV}{\mathcal{BV\!}}
\newcommand{\ABV}{\mathcal{ABV\!}}



%%%%%%%%%%    Math operators    %%%%%%%%%%%%%%%%%%%%%%%%%%%


\newcommand{\dif}[1]{\,\mathrm{d} #1}
%\newcommand{\norm}[1]{\lVert #1 \rVert}
%\newcommand{\abs}[1]{\left| #1 \right|}
\newcommand{\bnorm}[1]{\left\lVert #1\right\rVert}
\newcommand{\vii}[2]{\ensuremath{\begin{bmatrix}#1 \\ #2 \end{bmatrix}}}
\newcommand{\mii}[4]{\ensuremath{\begin{bmatrix}#1&#2 \\ #3&#4 \end{bmatrix}}}
\newcommand{\mc}[1]{\mathcal{#1}}

\newcommand{\one}{\mathds{1}}
\newcommand{\bigO}{\mathcal{O}}
\newcommand{\symmDiff}{\bigtriangleup} % \bigtriangleup


\DeclareMathOperator{\Image}{Image}
\DeclareMathOperator{\Vspan}{Span}
\DeclareMathOperator{\Erf}{erf}
\DeclareMathOperator{\Id}{Id}             % identity morphism
% \DeclareMathOperator{\ker}{ker}           % kernel
\DeclareMathOperator{\rg}{rg}             % image
\DeclareMathOperator{\defekt}{def}             % defect
\DeclareMathOperator{\im}{im}             % image
\DeclareMathOperator{\Hom}{Hom}           % homomorphisms
\DeclareMathOperator{\End}{End}           % endomorphisms
\DeclareMathOperator{\Span}{Span}         % linear span
\DeclareMathOperator{\grad}{\nabla}         % gradient
\DeclareMathOperator{\diam}{diam}         % gradient
\DeclareMathOperator{\Tr}{Tr}       	  % trace
\DeclareMathOperator{\diver}{Div}			% divergence
\DeclareMathOperator{\supp}{supp}			% support
\DeclareMathOperator{\dist}{dist}			% distance
\DeclareMathOperator{\inter}{int}			% interior
\DeclareMathOperator{\epi}{epi}			% epigraph
\DeclareMathOperator{\hyp}{hyp}			% hypograph
\DeclareMathOperator{\Lip}{Lip}			% lipschitz konstant
\DeclareMathOperator{\graph}{graph}			% graph
\DeclareMathOperator{\sgn}{sgn}			% sign
\DeclareMathOperator{\BMO}{BMO}			% BMO
\DeclareMathOperator{\mean}{mean}			% BMO
\DeclareMathOperator{\prox}{prox}
\DeclareMathOperator{\closure}{cl}
\DeclareMathOperator{\exterior}{ext}
\DeclareMathOperator{\interior}{int}
%\DeclareMathOperator{\B}{B}			% BMO


% \vect{ x // y // z } for a column vector with entries x, y, z
% similarly for larger vectors
% in this code:  1 = number of arguments
%               #1 = first argument
\newcommand{\vect}[1]{\begin{bmatrix} #1 \end{bmatrix}}

% \conj{z} for complex conjugation
\newcommand{\conj}{\overline}

%counter of current constant number:    
\newcounter{constant} 
%defines a new constant, but does not typeset anything:
\newcommand{\newconstant}[1]{\refstepcounter{constant}\label{#1}} 
%typesets named constant:
\newcommand{\useconstant}[1]{c_{\ref{#1}}}



%%%%%%% GENERAL STYLE %%%%%%%%%%%%%%%%%%

\setcounter{tocdepth}{3}
\setcounter{secnumdepth}{0}


%%%%%%% COLORS %%%%%%%%%%%%%%%%%%%%%%%%


\newcommand{\black}{\color{black}}


%%%%%% TITLE PAGE
%
%\subject{Specialised Course in Integration Theory, VT23}
%\title{Assignment Chapter 3.5}
%\author{Theo Koppenhöfer}
%\date{\today}
%
%
%%%%%% The content starts here %%%%%%%%%%%%%
%
%
%\begin{document}
%
%\maketitle
%
%
%%\nocite{*}
%\printbibliography
%
%\end{document}


%%%%%%% GENERAL STYLE %%%%%%%%%%%%%%%%%%
%
\setcounter{tocdepth}{1}
%\setcounter{secnumdepth}{0}


% modify beamer template
\setbeamercolor{footline}{fg=blue}
\setbeamerfont{footline}{size={\fontsize{10}{12}}}
\setbeamertemplate{navigation symbols}{}

%\setbeamertemplate{navigation symbols}{%
%    \usebeamerfont{footline}%
%    \usebeamercolor[fg]{footline}%
%    \hspace{1em}%
%    \raisebox{4pt}[0pt][0pt]{\insertframenumber/\inserttotalframenumber}
%}

%\setbeamertemplate{footline}{
%	\vspace*{0.1cm}
%	\hspace*{0.01cm}
%	\insertsection
%	\hfill\insertframenumber/\inserttotalframenumber
%	\hspace*{0.1cm}
%}

\setbeamertemplate{footline}[page number]

% modify handling of bibliography
\setbeamertemplate{bibliography entry title}{}
\setbeamertemplate{bibliography entry location}{}
\setbeamertemplate{bibliography entry note}{}

\setbeamertemplate{bibliography item}[text]


%%%%% allow for proofs over multiple slides

\makeatletter
\newenvironment<>{proofs}[1][\proofname]{%
    \par
    \def\insertproofname{#1\@addpunct{.}}%
    \usebeamertemplate{proof begin}#2}
  {\usebeamertemplate{proof end}}
\makeatother


%%%%%%% theorem environments

\newtheorem{proposition}{Proposition}

%%%%%% TITLE PAGE
%
%\subject{Specialised Course in Integration Theory, VT23}
%\title{Assignment Chapter 3.5}
%\author{Theo Koppenhöfer}
%\date{\today}
%
%
%%%%%% The content starts here %%%%%%%%%%%%%
%
%
%\begin{document}
%
%\maketitle
%
%
%%\nocite{*}
%\printbibliography
%
%\end{document}


%%%%% TITLE PAGE

\subject{Specialised course in integration theory, VT23}
\title{The Malý-Pfeffer integral}
\author{Theo Koppenhöfer}
\date{Lund \\[1ex] \today}


%\SetAlFnt{\small}
\addbibresource{../Presentation/bibliography.bib}

% environments: block, problem, proof, proofs, theorem, proposition, lemma, definition

\begin{document}

%\frame[plain]



% Frame 2
\frame[plain]{\titlepage}

% Frame 3
%\frame[plain]{ \frametitle{Table of contents} \tableofcontents }

\section{Introduction}
\begin{frame}
	Given a suitable set $A\subseteq\R^n$ and suitable $w\colon\R^n\to\R^n$ the divergence theorem states that
	\begin{align*}
		\int_A\diver w\dif\cL^n=\int_{\partial A}w\cdot\nu_A\dif\cH^{n-1}
	\end{align*}
	Here $\nu_A\colon \partial A\to S^{n-1}\subseteq\R^n$ is the exteriour unit normal and $\diver$ denotes the divergence.
\end{frame}

\section{Definition of the integral}

\begin{frame}
	\begin{definition}[essential interiour, exteriour, boundary]
	We calle the set of density points in $A$ essential interiour $\interiour_*A$ of $A$.
	The essential exteriour $\exteriour_*A=\interiour_*A^\complement$ is the essential interiour of the complement of $A$. The essential boundary is given by
	\begin{align*}
		\partial_*A=\R^n\setminus(\interiour_*A\cup\exteriour_*A)\,.
	\end{align*}
	\end{definition}
\end{frame}

\begin{frame}
	\begin{definition}[relative perimiter]
	We define the relative perimiter of a measurable set $E$ to be 
	\begin{align*}
		P(E,\text{in }A)=\cH^{n-1}\brk*{\partial_*E\cap \interiour_*A}\,.
	\end{align*}
	where $\cH^{n-1}$ denotes the $(n-1)$-dimensional Hausdorff-measure. For convenience we write $$P(E)=P(E,\text{in }\R^n)\,.$$
	\end{definition}
\end{frame}

\begin{frame}
	\begin{definition}[$\BV$-sets]
	A measurable set $A\subseteq\R^n$ is called a $\BV$-set if $ \abs{A}+P(A)<\infty$. We denote by $\BV$ the set of all $\BV$-sets and by $\BV_c$ the set of all bounded $\BV$-sets.
	\end{definition}
	\begin{definition}[Topology on $\BV_c$]
	We say a sequence $A\colon\N\to \BV_c$ converges to $A_*$ if there exists a compact $K\subseteq\R^n$ such that $A_k\subseteq K$, $\sup_kP(A_k)<\infty$ and $\abs{A_*\symmDiff A_k}\to0$ as $k\to\infty$. Here $A\symmDiff B=\brk*{B\setminus A}\sqcup \brk*{A\setminus B}$ denotes the symmetric difference.
	\end{definition}
\end{frame}

\begin{frame}
	\begin{definition}[Charge]
	A function $F\colon \BV_c\to\R$ is called
	\begin{itemize}
		\item finitely additive if $F(A\sqcup B) =F(A)+F(B)$ for all disjoint $A,B\in\BV_c$.
		\item continuous if $A_k\to A_*$ implies that $F(A_k)\to F(A_*)$.
		\item a charge if it is finitely additive and continuous
	\end{itemize}
	\end{definition}
\end{frame}

\begin{frame}
	\begin{Example}[Indefinite Lebesgue-Integrals are charges]
	Let $f\in L^1_{loc}(\R^n)$ then the indefinite integral of $f$
	\begin{align*}
		F\colon A\mapsto \int_Af\dif\cL^n
	\end{align*}
	is a charge.
	\end{Example}
	\begin{Example}[Fluxes are charges]
	For $E\in\BV$  and $w\in C(\closure(E);\R^n)$ we have that the flux of $w$
	\begin{align*}
		F\colon A\mapsto \int_{\partial_*(A\cap E)}w\cdot\nu_{A\cap E}\dif \cH^{n-1}
	\end{align*}
	is a charge. Here $\nu_{A\cap E}\colon \partial_*(A\cap E)\to S^{n-1}\subseteq\R^n$ denotes the outer unit normal.
	\end{Example}
\end{frame}

\begin{frame}
	\begin{definition}[Regularity of $\BV_c$-sets]
	For $E\in \BV_c$ we define the regularity
	\begin{align*}
		r(E)=\begin{cases}
			\frac{\abs{E}}{\diam(E)P(E)} &\text{ if }\abs{E}>0 \\
			0 &\text{ else}
		\end{cases}
	\end{align*}
	\end{definition}
	\begin{definition}[$\e$-isoparametric]
	We call $E\in\BV_c$ $\e$-isoparametric if for all $T\in\BV$
	\begin{align*}
		\min\brk[c]*{P(E\cap T),P(E\setminus T)}\leq \frac{P(T,\text{in }E)}{\e}\,.
	\end{align*}
	\end{definition}
	\begin{definition}[Gauge]
	We call a set thin if it is $\sigma$-finite w.r.t.\ $\cH^{n-1}$. A mapping $\delta\colon\R^n\to\R^n_{>0}$ for which $\brk[c]{\delta=0}$ is thin is called a gauge.
	\end{definition}
\end{frame}

\begin{frame}
	\begin{definition}[Partitions]
	Let $\delta$ be a gauge and $\e>0$.
	We call
	\begin{align*}
		\cP=\brk[c]*{(E_1,x_1),\dots, (E_p,x_p)}
	\end{align*}
	a partition of the set $\bigcup\cP=\bigcup_iE_i$ if $E_i\in\BV_c$ are disjoint sets and $x_i\in\R^n$. A partition is called
	\begin{itemize}
		\item dyadic if $E_i$ is a dyadic cube and $x_i\in E_i$ for all $i$
		\item $\e$-regular if $r(E_i\cup\brk[c]{x_i})>\e$ for all $i$
		\item strongly $\e$-regular if it is $\e$-regular, $E_i$ is $\e$-isoperimetric and $x_i\in\closure_*E_i$ for all $i$
		\item $\delta$-fine if $E_i\subseteq B_{\delta(x_i)}(x_i)$ for all $i$
	\end{itemize}
	\end{definition}
\end{frame}

\begin{frame}
	\begin{definition}[$R^*$-integral]
	A function $f\colon \R^n\to\R$ is called $R^*$-integrable with respect to a charge $G$ if there is a charge $F$, s.t.\ for all $\e>0$ there exists a gauge $\delta$ such that
	\begin{align*}
		\sum_{i=1}^p\abs{f(x_i)G(E_i)-F(E_i)}<\e
	\end{align*}
	for each strongly $\e$-regular $\delta$-fine partition $\brk[c]*{(E_1,x_1),\dots,(E_p,x_p)}$. We call $F$ an indefinite integral of $f$ with respect to $G$ and write
	\begin{align*}
		F=(R^*)\int f\dif G\,.
	\end{align*}
	\end{definition}
\end{frame}


\section{Uniqueness of the integral}
\begin{frame}
	\begin{proposition}
	The integral is unique.
	\end{proposition}
\end{frame}

\begin{frame}
	\begin{proposition}[Linearity of the integral]
	Let $f_1,f_2$ be $R_*$-integrable and $a\in\R$. Then $f_1+af_2$ is also integrable and
	\begin{align*}
		(R^*)\int f_1\dif G + (R^*)\int f_2\dif G=(R^*)\int f_1+\alpha f_2\dif G\,.
	\end{align*}
	\end{proposition}
\end{frame}

\section{What the integral is good for}
\begin{frame}
	\begin{proposition}[Generalisation of the Lebesgue integral on $\R^n$]
	Each Lebesgue-integrable function is also $R_*$-integrable and the integrals conincide.
	\end{proposition}
	\begin{proof}
	See \cite[Proposition 3.5]{Pfe2016}.
	\end{proof}
\end{frame}

\begin{frame}
	\begin{proposition}[Generalisation of the Henstock-Kurzweil integral on $\R$]
	A function $f\colon\R\to\R$ is Denjoy-Perron integrable on a compact $A\subseteq\R$ iff it is $R_*$-integrable and the two integrals coincide on $A$.
	\end{proposition}
	\begin{proof}
	See \cite[Proposition 3.6]{Pfe2016}.
	\end{proof}
\end{frame}

\begin{frame}
	\begin{definition}[Admissable sets]
	We call a set admissable if $\interiour_*A\subseteq A\subseteq \closure_*A$ and $\partial A$ is compact. The set of admissible $\BV$-sets is denoted by $\ABV$.
	\end{definition}
	
	\begin{proposition}[Generalisation of the Pfeffer-Integral on $\R^n$]\label{pr:GeneralisationPfeffer}
	Let $A\in\ABV$. Then each Pfeffer-integrable function is also $R_*$-integrable and the integrals coincide on $A$.
	\end{proposition}
\end{frame}

\begin{frame}
	\begin{theorem}[Divergence theorem]
	Let $A\in\ABV$, $S\subseteq A$ a thin set and $w\in C(\closure(A);\R^n)$ a continuous vector field which is pointwise Lipschitz on $A\setminus S$. Then $\diver w$ is $R_*$-integrable and
	\begin{align*}
		(R^*)\int_A\diver w\dif\cL^n=\int_{\partial_* A}w\cdot \nu_A\dif \cH^{n-1}
	\end{align*}
	where $\nu_A\colon \partial_* A\to S^{n-1}\subseteq\R^n$ is the unit normal to $A$.
	\end{theorem}
\end{frame}


\section{Summary}

\section*{Sources}

\nocite{*}

\begin{frame}
	\frametitle{Main source}
%	\bibliographystyle{plain}
%	\bibliography{bibliography}
	\printbibliography[title={Main source},keyword={main}]
\end{frame}

\begin{frame}[allowframebreaks]
	\frametitle{Other sources}
	\printbibliography[keyword={secondary}, title={Other sources}]
\end{frame}

\begin{frame}[plain]
	\begin{center}
		\Large{{Thank you for your attention.}}
	\end{center}
\end{frame}

%\frame[plain]

\end{document}
