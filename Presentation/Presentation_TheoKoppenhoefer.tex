%

\input{../Latex_Templates/Preamble_Presentation}

%%%%% TITLE PAGE

\subject{Specialised course in integration theory, VT23}
\title{The Malý-Pfeffer integral}
\author{\url{https://github.com/TheoKoppenhoefer/integration-
theory-VT23}}
\date{Lund \\[1ex] \today}


%\SetAlFnt{\small}
\addbibresource{../Presentation/bibliography.bib}

% environments: block, problem, proof, proofs, theorem, proposition, lemma, definition

\begin{document}

%\frame[plain]



% Frame 2
\frame[plain]{\titlepage}

% Frame 3
\frame[plain]{ \frametitle{Table of contents} \tableofcontents }

\section{Introduction}
\begin{frame}
	\frametitle{Introduction}
	Given a suitable set $A\subseteq\R^n$ and suitable $w\colon\R^n\to\R^n$ the divergence theorem states that
	\begin{align*}
		\int_A\diver w\dif\cL^n=\int_{\partial A}w\cdot\nu_A\dif\cH^{n-1}
	\end{align*}
	Here $\nu_A\colon \partial A\to S^{n-1}\subseteq\R^n$ is the exteriour unit normal and $\diver$ denotes the divergence.
	\begin{itemize}
		\item 
		Generalising the LHS leads to a formulation involving the Pfeffer integral which is defined in \cite{Pfe1991}.
		\item
		The Pfeffer (and the Henstock-Kurzweil) integral was generalised in \cite{Pfe2016} which I call the Malý-Pfeffer integral. It is the topic of this presentation.
	\end{itemize}
\end{frame}

\section{Definition of the integral}

\begin{frame}
	\frametitle{Definition of the integral}
	\begin{definition}[essential interiour, exteriour, boundary]
	We call the set of density points of $A$ essential interiour $\interiour_*A$ of $A$.
	The essential exteriour $\exteriour_*A=\interiour_*A^\complement$ is the essential interiour of the complement of $A$. The essential boundary is given by
	\begin{align*}
		\partial_*A=\R^n\setminus(\interiour_*A\cup\exteriour_*A)\,.
	\end{align*}
	\end{definition}
\end{frame}

\begin{frame}
	\begin{definition}[relative perimiter]
	We define the relative perimeter of a measurable set $E$ to be 
	\begin{align*}
		P(E,\text{in }A)=\cH^{n-1}\brk*{\partial_*E\cap \interiour_*A}\,.
	\end{align*}
	where $\cH^{n-1}$ denotes the $(n-1)$-dimensional Hausdorff-measure. For convenience we write $$P(E)=P(E,\text{in }\R^n)\,.$$
	\end{definition}
\end{frame}

\begin{frame}
	\begin{definition}[$\BV$-sets]
	A measurable set $A\subseteq\R^n$ is called a $\BV$-set if $ \abs{A}+P(A)<\infty$. We denote by $\BV$ the set of all $\BV$-sets and by $\BV_c$ the set of all bounded $\BV$-sets.
	\end{definition}
	\begin{definition}[Topology on $\BV_c$]
	We say a sequence $A\colon\N\to \BV_c$ converges to $A_*$ if there exists a compact $K\subseteq\R^n$ such that $A_k\subseteq K$, $\sup_kP(A_k)<\infty$ and $\abs{A_*\symmDiff A_k}\to0$ as $k\to\infty$. Here $A\symmDiff B=\brk*{B\setminus A}\sqcup \brk*{A\setminus B}$ denotes the symmetric difference.
	\end{definition}
\end{frame}

\begin{frame}
	\begin{definition}[Charge]
	A function $F\colon \BV_c\to\R$ is called
	\begin{itemize}
		\item finitely additive if $F(A\sqcup B) =F(A)+F(B)$ for all disjoint $A,B\in\BV_c$.
		\item continuous if $A_k\to A_*$ implies that $F(A_k)\to F(A_*)$.
		\item a charge if it is finitely additive and continuous
	\end{itemize}
	\end{definition}
	\begin{Example}[Indefinite Lebesgue-Integrals are charges]
	Let $f\in L^1_{loc}(\R^n)$ then the indefinite integral of $f$
	\begin{align*}
		F\colon A\mapsto \int_Af\dif\cL^n
	\end{align*}
	is a charge.
	\end{Example}
\end{frame}

\begin{frame}
	\begin{definition}[Charge]
	A function $F\colon \BV_c\to\R$ is called
	\begin{itemize}
		\item finitely additive if $F(A\sqcup B) =F(A)+F(B)$ for all disjoint $A,B\in\BV_c$.
		\item continuous if $A_k\to A_*$ implies that $F(A_k)\to F(A_*)$.
		\item a charge if it is finitely additive and continuous
	\end{itemize}
	\end{definition}
	\begin{Example}[Fluxes are charges]
	For $E\in\BV$  and $w\in C(\closure(E);\R^n)$ we have that the flux of $w$
	\begin{align*}
		F\colon A\mapsto \int_{\partial_*(A\cap E)}w\cdot\nu_{A\cap E}\dif \cH^{n-1}
	\end{align*}
	is a charge. Here $\nu_{A\cap E}\colon \partial_*(A\cap E)\to S^{n-1}\subseteq\R^n$ denotes the outer unit normal.
	\end{Example}
\end{frame}

\begin{frame}
	\begin{definition}[Regularity of $\BV_c$-sets]
	For $E\in \BV_c$ we define the regularity
	\begin{align*}
		r(E)=\begin{cases}
			\frac{\abs{E}}{\diam(E)P(E)} &\text{ if }\abs{E}>0 \\
			0 &\text{ else}
		\end{cases}
	\end{align*}
	\end{definition}
	\begin{definition}[$\e$-isoperimetric]
	We call $E\in\BV_c$ $\e$-isoperimetric if for all $T\in\BV$
	\begin{align*}
		\min\brk[c]*{P(E\cap T),P(E\setminus T)}\leq \frac{P(T,\text{in }E)}{\e}\,.
	\end{align*}
	\end{definition}
	\begin{definition}[Gauge]
	We call a set thin if it is $\sigma$-finite w.r.t.\ $\cH^{n-1}$. A mapping $\delta\colon\R^n\to\R_{\geq 0}$ for which $\brk[c]{\delta=0}$ is thin is called a gauge.
	\end{definition}
\end{frame}

\begin{frame}
	\begin{definition}[Partitions]
	Let $\delta$ be a gauge and $\e>0$.
	We call
	\begin{align*}
		\cP=\brk[c]*{(E_1,x_1),\dots, (E_p,x_p)}
	\end{align*}
	a partition of the set $\bigcup\cP=\bigcup_iE_i$ if $E_i\in\BV_c$ are disjoint sets and $x_i\in\R^n$. A partition is called
	\begin{itemize}
		\item dyadic if $E_i$ is a dyadic cube and $x_i\in \closure(E_i)$ for all $i$
		\item $\e$-regular if $r(E_i\cup\brk[c]{x_i})>\e$ for all $i$
		\item strongly $\e$-regular if it is $\e$-regular, $E_i$ is $\e$-isoperimetric and $x_i\in\closure_*E_i$ for all $i$
		\item $\delta$-fine if $E_i\subseteq B_{\delta(x_i)}(x_i)$ for all $i$
	\end{itemize}
	\end{definition}
\end{frame}

\begin{frame}
	\begin{definition}[$R_*$-integral]
	A function $f\colon \R^n\to\R$ is called $R_*$-integrable with respect to a charge $G$ if there is a charge $F$, s.t.\ for all $\e>0$ there exists a gauge $\delta$ such that
	\begin{align*}
		\sum_{i=1}^p\abs{f(x_i)G(E_i)-F(E_i)}<\e
	\end{align*}
	for each strongly $\e$-regular $\delta$-fine partition $\brk[c]*{(E_1,x_1),\dots,(E_p,x_p)}$. We call $F$ an indefinite integral of $f$ with respect to $G$ and write
	\begin{align*}
		F=(R_*)\!\!\int f\dif G\,.
	\end{align*}
	\end{definition}
\end{frame}


\section{Uniqueness and linearity}
\begin{frame}
	\frametitle{Uniqueness and linearity}
	\begin{definition}[Nice dyadic cubes]
	Let $\sigma\colon \closure(C)\to\R_{>0}$. A dyadic cube $C$ is called nice if there exist a dyadic $\sigma$-fine partition of $C$.
	A dyadic cube which is not nice is called faulty.
	\end{definition}
	
	\begin{lemma}[Cousin]\label{le:Cousin}
	All dyadic cubes are nice.
	\end{lemma}
\end{frame}

\begin{frame}
	\begin{proof}
	Assume a dyadic cube $C=C^1$ is faulty and has diameter $r$. Then $C^1$ can be written as the disjoint union $C=\bigsqcup_iC_i$ of dyadic cubes $C_i$ with diameters less than $r/2$. Since $C$ is faulty at least one of the $C_i$, say $C^2=C_i$ is also faulty. Inductively we obtain a sequence of nested faulty dyadic cubes $C^j$ with diameters less than $r/2^j$. Thus
	\begin{align*}
		\bigcap_j\closure\brk*{C^j}=\brk[c]{x}
	\end{align*}
	for some $x\in \closure(C)$. Let $j$ be s.t.\ $r/2^j<\sigma(x)$. Then we have that $\diam(C^j)<\sigma(x)$ and $x\in C^j$ so $C^j$ is nice. This is a contradiction.
	\end{proof}
\end{frame}

\begin{frame}
	One uses this to prove the following result
	\begin{lemma}[Almost covering of a cube]\label{le:DisjointCube}
	Let $C$ be a dyadic cube, $F$ be a charge, $\e>0$ and $\delta$ be a gauge. Then there exists a $\delta$-fine dyadic partition $\cP=\brk[c]*{(C_1,x_1),\dots,(C_q,x_q)}$ such that
	\begin{align*}
		\abs{F}\brk*{C\setminus\bigcup\cP}<\e\,.
	\end{align*}
	\end{lemma}
\end{frame}

\begin{frame}
	\begin{proposition}
	The integral is unique.
	\end{proposition}
	\begin{proof}
	\vspace*{0.5cm}
	\hspace*{1cm}
	\centering
	\scalebox{0.8}{
	\centering
	% Graphic for TeX using PGF
% Title: /mnt/64BB-4184/Filing/Education/Lund/Courses/IntegrationTheory/integration-theory-VT23/Figures/UniquenessProof.dia
% Creator: Dia v0.97+git
% CreationDate: Tue May 30 14:29:20 2023
% For: theo
% \usepackage{tikz}
% The following commands are not supported in PSTricks at present
% We define them conditionally, so when they are implemented,
% this pgf file will use them.
\ifx\du\undefined
  \newlength{\du}
\fi
\setlength{\du}{15\unitlength}
\begin{tikzpicture}[even odd rule]
\pgftransformxscale{1.000000}
\pgftransformyscale{-1.000000}
\definecolor{dialinecolor}{rgb}{0.000000, 0.000000, 0.000000}
\pgfsetstrokecolor{dialinecolor}
\pgfsetstrokeopacity{1.000000}
\definecolor{diafillcolor}{rgb}{1.000000, 1.000000, 1.000000}
\pgfsetfillcolor{diafillcolor}
\pgfsetfillopacity{1.000000}
\pgfsetlinewidth{0.100000\du}
\pgfsetdash{}{0pt}
\pgfsetmiterjoin
{\pgfsetcornersarced{\pgfpoint{0.000000\du}{0.000000\du}}\definecolor{diafillcolor}{rgb}{1.000000, 1.000000, 1.000000}
\pgfsetfillcolor{diafillcolor}
\pgfsetfillopacity{1.000000}
\fill (16.586605\du,5.200000\du)--(16.586605\du,7.100000\du)--(22.931605\du,7.100000\du)--(22.931605\du,5.200000\du)--cycle;
}{\pgfsetcornersarced{\pgfpoint{0.000000\du}{0.000000\du}}\definecolor{dialinecolor}{rgb}{0.000000, 0.000000, 0.000000}
\pgfsetstrokecolor{dialinecolor}
\pgfsetstrokeopacity{1.000000}
\draw (16.586605\du,5.200000\du)--(16.586605\du,7.100000\du)--(22.931605\du,7.100000\du)--(22.931605\du,5.200000\du)--cycle;
}% setfont left to latex
\definecolor{dialinecolor}{rgb}{0.000000, 0.000000, 0.000000}
\pgfsetstrokecolor{dialinecolor}
\pgfsetstrokeopacity{1.000000}
\definecolor{diafillcolor}{rgb}{0.000000, 0.000000, 0.000000}
\pgfsetfillcolor{diafillcolor}
\pgfsetfillopacity{1.000000}
\node[anchor=base,inner sep=0pt, outer sep=0pt,color=dialinecolor] at (19.759105\du,6.344053\du){Cousin's Lemma};
\pgfsetlinewidth{0.100000\du}
\pgfsetdash{}{0pt}
\pgfsetmiterjoin
{\pgfsetcornersarced{\pgfpoint{0.000000\du}{0.000000\du}}\definecolor{diafillcolor}{rgb}{1.000000, 1.000000, 1.000000}
\pgfsetfillcolor{diafillcolor}
\pgfsetfillopacity{1.000000}
\fill (15.321250\du,9.100000\du)--(15.321250\du,11.000000\du)--(24.178750\du,11.000000\du)--(24.178750\du,9.100000\du)--cycle;
}{\pgfsetcornersarced{\pgfpoint{0.000000\du}{0.000000\du}}\definecolor{dialinecolor}{rgb}{0.000000, 0.000000, 0.000000}
\pgfsetstrokecolor{dialinecolor}
\pgfsetstrokeopacity{1.000000}
\draw (15.321250\du,9.100000\du)--(15.321250\du,11.000000\du)--(24.178750\du,11.000000\du)--(24.178750\du,9.100000\du)--cycle;
}% setfont left to latex
\definecolor{dialinecolor}{rgb}{0.000000, 0.000000, 0.000000}
\pgfsetstrokecolor{dialinecolor}
\pgfsetstrokeopacity{1.000000}
\definecolor{diafillcolor}{rgb}{0.000000, 0.000000, 0.000000}
\pgfsetfillcolor{diafillcolor}
\pgfsetfillopacity{1.000000}
\node[anchor=base,inner sep=0pt, outer sep=0pt,color=dialinecolor] at (19.750000\du,10.244053\du){Almost covering Lemma};
\pgfsetlinewidth{0.100000\du}
\pgfsetdash{}{0pt}
\pgfsetmiterjoin
{\pgfsetcornersarced{\pgfpoint{0.000000\du}{0.000000\du}}\definecolor{diafillcolor}{rgb}{1.000000, 1.000000, 1.000000}
\pgfsetfillcolor{diafillcolor}
\pgfsetfillopacity{1.000000}
\fill (19.323750\du,14.200000\du)--(19.323750\du,16.100000\du)--(29.076250\du,16.100000\du)--(29.076250\du,14.200000\du)--cycle;
}{\pgfsetcornersarced{\pgfpoint{0.000000\du}{0.000000\du}}\definecolor{dialinecolor}{rgb}{0.000000, 0.000000, 0.000000}
\pgfsetstrokecolor{dialinecolor}
\pgfsetstrokeopacity{1.000000}
\draw (19.323750\du,14.200000\du)--(19.323750\du,16.100000\du)--(29.076250\du,16.100000\du)--(29.076250\du,14.200000\du)--cycle;
}% setfont left to latex
\definecolor{dialinecolor}{rgb}{0.000000, 0.000000, 0.000000}
\pgfsetstrokecolor{dialinecolor}
\pgfsetstrokeopacity{1.000000}
\definecolor{diafillcolor}{rgb}{0.000000, 0.000000, 0.000000}
\pgfsetfillcolor{diafillcolor}
\pgfsetfillopacity{1.000000}
\node[anchor=base,inner sep=0pt, outer sep=0pt,color=dialinecolor] at (24.200000\du,15.344053\du){Uniqueness of the integral};
\pgfsetlinewidth{0.100000\du}
\pgfsetdash{}{0pt}
\pgfsetmiterjoin
{\pgfsetcornersarced{\pgfpoint{0.000000\du}{0.000000\du}}\definecolor{diafillcolor}{rgb}{1.000000, 1.000000, 1.000000}
\pgfsetfillcolor{diafillcolor}
\pgfsetfillopacity{1.000000}
\fill (25.323750\du,8.346967\du)--(25.323750\du,11.046967\du)--(34.073750\du,11.046967\du)--(34.073750\du,8.346967\du)--cycle;
}{\pgfsetcornersarced{\pgfpoint{0.000000\du}{0.000000\du}}\definecolor{dialinecolor}{rgb}{0.000000, 0.000000, 0.000000}
\pgfsetstrokecolor{dialinecolor}
\pgfsetstrokeopacity{1.000000}
\draw (25.323750\du,8.346967\du)--(25.323750\du,11.046967\du)--(34.073750\du,11.046967\du)--(34.073750\du,8.346967\du)--cycle;
}% setfont left to latex
\definecolor{dialinecolor}{rgb}{0.000000, 0.000000, 0.000000}
\pgfsetstrokecolor{dialinecolor}
\pgfsetstrokeopacity{1.000000}
\definecolor{diafillcolor}{rgb}{0.000000, 0.000000, 0.000000}
\pgfsetfillcolor{diafillcolor}
\pgfsetfillopacity{1.000000}
\node[anchor=base,inner sep=0pt, outer sep=0pt,color=dialinecolor] at (29.698750\du,9.491020\du){Density of dyadic cubes};
% setfont left to latex
\definecolor{dialinecolor}{rgb}{0.000000, 0.000000, 0.000000}
\pgfsetstrokecolor{dialinecolor}
\pgfsetstrokeopacity{1.000000}
\definecolor{diafillcolor}{rgb}{0.000000, 0.000000, 0.000000}
\pgfsetfillcolor{diafillcolor}
\pgfsetfillopacity{1.000000}
\node[anchor=base,inner sep=0pt, outer sep=0pt,color=dialinecolor] at (29.698750\du,10.291020\du){in $\BV_c$};
\pgfsetlinewidth{0.100000\du}
\pgfsetdash{}{0pt}
\pgfsetbuttcap
{
\definecolor{diafillcolor}{rgb}{0.000000, 0.000000, 0.000000}
\pgfsetfillcolor{diafillcolor}
\pgfsetfillopacity{1.000000}
% was here!!!
\pgfsetarrowsend{stealth}
\definecolor{dialinecolor}{rgb}{0.000000, 0.000000, 0.000000}
\pgfsetstrokecolor{dialinecolor}
\pgfsetstrokeopacity{1.000000}
\draw (19.756019\du,7.150018\du)--(19.750000\du,9.100000\du);
}
\pgfsetlinewidth{0.100000\du}
\pgfsetdash{}{0pt}
\pgfsetmiterjoin
\pgfsetbuttcap
{
\definecolor{diafillcolor}{rgb}{0.000000, 0.000000, 0.000000}
\pgfsetfillcolor{diafillcolor}
\pgfsetfillopacity{1.000000}
% was here!!!
\pgfsetarrowsend{stealth}
{\pgfsetcornersarced{\pgfpoint{0.000000\du}{0.000000\du}}\definecolor{dialinecolor}{rgb}{0.000000, 0.000000, 0.000000}
\pgfsetstrokecolor{dialinecolor}
\pgfsetstrokeopacity{1.000000}
\draw (19.750000\du,11.050488\du)--(19.750000\du,12.625244\du)--(24.200000\du,12.625244\du)--(24.200000\du,14.200000\du);
}}
\pgfsetlinewidth{0.100000\du}
\pgfsetdash{}{0pt}
\pgfsetmiterjoin
\pgfsetbuttcap
{
\definecolor{diafillcolor}{rgb}{0.000000, 0.000000, 0.000000}
\pgfsetfillcolor{diafillcolor}
\pgfsetfillopacity{1.000000}
% was here!!!
\pgfsetarrowsend{stealth}
{\pgfsetcornersarced{\pgfpoint{0.000000\du}{0.000000\du}}\definecolor{dialinecolor}{rgb}{0.000000, 0.000000, 0.000000}
\pgfsetstrokecolor{dialinecolor}
\pgfsetstrokeopacity{1.000000}
\draw (29.698750\du,11.046967\du)--(29.698750\du,12.623484\du)--(24.200000\du,12.623484\du)--(24.200000\du,14.200000\du);
}}
\end{tikzpicture}

	}
	\end{proof}
\end{frame}

\begin{frame}
	\begin{proposition}[Linearity of the integral]
	Let $f_1,f_2$ be $R_*$-integrable and $\alpha\in\R$. Then $f_1+\alpha f_2$ is also $R_*$-integrable and
	\begin{align*}
		(R_*)\!\!\int f_1\dif G + (R_*)\!\!\int \alpha f_2\dif G=(R_*)\!\!\int f_1+\alpha f_2\dif G\,.
	\end{align*}
	\end{proposition}
\end{frame}

\begin{frame}
	\begin{proof}
	We write $F_i=(R_*)\!\!\int f_i\dif G$. then we have that for all $\e>0$ there exist gauges $\delta_i$ such that
	\begin{align*}
		\sum_j\abs*{f_i(x_j)G(E_j)-F_i(E_j)}<\e
	\end{align*}
	for each strongly $\e$-regular $\delta_i$-fine partition $\brk[c]{(E_1,x_1),\dots,(E_p,x_p)}$. Since the space of charges is a linear space we have that also $F_1+\alpha F_2$ is a charge.
	If we now set $\delta=\min_i\delta_i$ then we obtain that
	\begin{align*}
		&\sum_j\abs*{(f_1+\alpha f_2)(x_j)G(E_j)-(F_1+\alpha F_2)(E_j)} \\
		&\leq \sum_j\abs*{f_1(x_j)G(E_j)-F_1(E_j)}+\abs{\alpha}\sum_j\abs*{f_2(x_j)G(E_j)-F_2(E_j)} \\
		&\leq (1+\abs{\alpha})\e
	\end{align*}
	for every strongly $\e$-regular $\delta$-fine partition $\brk[c]*{(E_1,x_1),\dots,(E_p,x_p)}$. Thus $f_1+\alpha f_2$ is integrable with integral $F_1+\alpha F_2$.
	\end{proof}
\end{frame}

\section{What this integral is good for}
\begin{frame}
	\frametitle{What this integral is good for}
	\begin{proposition}[Generalisation of the Lebesgue integral on $\R^n$]
	Each Lebesgue-integrable function is also $R_*$-integrable and the integrals coincide.
	\end{proposition}
	\begin{proof}
	See \cite[Proposition 3.5]{Pfe2016}.
	\end{proof}
\end{frame}

\begin{frame}
	\begin{proposition}[Generalisation of the Henstock-Kurzweil integral on $\R$]
	A function $f\colon\R\to\R$ is Henstock-Kurzweil integrable on a compact $A\subseteq\R$ iff it is $R_*$-integrable and the two integrals coincide on $A$.
	\end{proposition}
	\begin{proof}
	See \cite[Proposition 3.6]{Pfe2016}.
	\end{proof}
\end{frame}

\begin{frame}
	\begin{definition}[Admissable sets]
	We call a set admissible if $\interiour_*A\subseteq A\subseteq \closure_*A$ and $\partial A$ is compact. The set of admissible $\BV$-sets is denoted by $\ABV$.
	\end{definition}
	
	\begin{proposition}[Generalisation of the Pfeffer-Integral on $\R^n$]\label{pr:GeneralisationPfeffer}
	Let $A\in\ABV$. Then each Pfeffer-integrable function is also $R_*$-integrable and the integrals coincide on $A$.
	\end{proposition}
\end{frame}

\begin{frame}
	\begin{theorem}[Divergence theorem]
	Let $A\in\ABV$, $S\subseteq A$ a thin set and $w\in C(\closure(A);\R^n)$ a continuous vector field which is point-wise Lipschitz on $A\setminus S$. Then $\diver w$ is $R_*$-integrable and
	\begin{align*}
		(R_*)\!\!\int_A\diver w\dif\cL^n=\int_{\partial_* A}w\cdot \nu_A\dif \cH^{n-1}
	\end{align*}
	where $\nu_A\colon \partial_* A\to S^{n-1}\subseteq\R^n$ is the unit normal to $A$.
	\end{theorem}
\end{frame}


\section{Summary}
\begin{frame}
	\frametitle{Summary}
	\begin{itemize}
		\item The construction is very similar to that of the Henstock-Kurzweil integral and involves $\delta$-fine partitions where $\delta$ is a gauge
		\item One can prove that the integral is unique (using Cousin's Lemma and an 'almost covering Lemma')
		\item The $R_*$-integral generalises the Pfeffer and the Lebesgue integral on $\R^n$
		\item It generalises the Henstock-Kurzweil integral on $\R$
		\item One can formulate a very general version of the divergence theorem for this integral
	\end{itemize}
\end{frame}

\section*{Sources}

\nocite{*}

\begin{frame}
	\frametitle{Main source}
%	\bibliographystyle{plain}
%	\bibliography{bibliography}
	\printbibliography[title={Main source},keyword={main}]
\end{frame}

\begin{frame}[allowframebreaks]
	\frametitle{Other sources}
	\printbibliography[keyword={secondary}, title={Other sources}]
\end{frame}

\begin{frame}[plain]
	\begin{center}
		\Large{{Thank you for your attention.}}
	\end{center}
\end{frame}

%\frame[plain]

\end{document}
