%

\input{../Latex_Templates/Preamble_Presentation}

%%%%% TITLE PAGE

\subject{Specialised course in integration theory, VT23}
\title{The Malý-Pfeffer integral}
\author{Theo Koppenhöfer}
\date{Lund \\[1ex] \today}


%\SetAlFnt{\small}
\addbibresource{../Presentation/bibliography.bib}

% environments: block, problem, proof, proofs, theorem, proposition, lemma, definition

\begin{document}

%\frame[plain]



% Frame 2
\frame[plain]{\titlepage}

% Frame 3
%\frame[plain]{ \frametitle{Table of contents} \tableofcontents }

\section{Introduction}
\begin{frame}
	Given a suitable set $A\subseteq\R^n$ and suitable $w\colon\R^n\to\R^n$ the divergence theorem states that
	\begin{align*}
		\int_A\diver w\dif\cL^n=\int_{\partial A}w\cdot\nu_A\dif\cH^{n-1}
	\end{align*}
	Here $\nu_A\colon \partial A\to S^{n-1}\subseteq\R^n$ is the exteriour unit normal and $\diver$ denotes the divergence.
\end{frame}

\section{Definition of the integral}

\begin{frame}
	\begin{definition}[essential interiour, exteriour, boundary]
	We calle the set of density points in $A$ essential interiour $\interiour_*A$ of $A$.
	The essential exteriour $\exteriour_*A=\interiour_*A^\complement$ is the essential interiour of the complement of $A$. The essential boundary is given by
	\begin{align*}
		\partial_*A=\R^n\setminus(\interiour_*A\cup\exteriour_*A)\,.
	\end{align*}
	\end{definition}
\end{frame}

\begin{frame}
	\begin{definition}[relative perimiter]
	We define the relative perimiter of a measurable set $E$ to be 
	\begin{align*}
		P(E,\text{in }A)=\cH^{n-1}\brk*{\partial_*E\cap \interiour_*A}\,.
	\end{align*}
	where $\cH^{n-1}$ denotes the $(n-1)$-dimensional Hausdorff-measure. For convenience we write $$P(E)=P(E,\text{in }\R^n)\,.$$
	\end{definition}
\end{frame}

\begin{frame}
	\begin{definition}[$\BV$-sets]
	A measurable set $A\subseteq\R^n$ is called a $\BV$-set if $ \abs{A}+P(A)<\infty$. We denote by $\BV$ the set of all $\BV$-sets and by $\BV_c$ the set of all bounded $\BV$-sets.
	\end{definition}
	\begin{definition}[Topology on $\BV_c$]
	We say a sequence $A\colon\N\to \BV_c$ converges to $A_*$ if there exists a compact $K\subseteq\R^n$ such that $A_k\subseteq K$, $\sup_kP(A_k)<\infty$ and $\abs{A_*\symmDiff A_k}\to0$ as $k\to\infty$. Here $A\symmDiff B=\brk*{B\setminus A}\sqcup \brk*{A\setminus B}$ denotes the symmetric difference.
	\end{definition}
\end{frame}

\begin{frame}
	\begin{definition}[Charge]
	A function $F\colon \BV_c\to\R$ is called
	\begin{itemize}
		\item finitely additive if $F(A\sqcup B) =F(A)+F(B)$ for all disjoint $A,B\in\BV_c$.
		\item continuous if $A_k\to A_*$ implies that $F(A_k)\to F(A_*)$.
		\item a charge if it is finitely additive and continuous
	\end{itemize}
	\end{definition}
\end{frame}

\begin{frame}
	\begin{Example}[Indefinite Lebesgue-Integrals are charges]
	Let $f\in L^1_{loc}(\R^n)$ then the indefinite integral of $f$
	\begin{align*}
		F\colon A\mapsto \int_Af\dif\cL^n
	\end{align*}
	is a charge.
	\end{Example}
	\begin{Example}[Fluxes are charges]
	For $E\in\BV$  and $w\in C(\closure(E);\R^n)$ we have that the flux of $w$
	\begin{align*}
		F\colon A\mapsto \int_{\partial_*(A\cap E)}w\cdot\nu_{A\cap E}\dif \cH^{n-1}
	\end{align*}
	is a charge. Here $\nu_{A\cap E}\colon \partial_*(A\cap E)\to S^{n-1}\subseteq\R^n$ denotes the outer unit normal.
	\end{Example}
\end{frame}

\begin{frame}
	\begin{definition}[Regularity of $\BV_c$-sets]
	For $E\in \BV_c$ we define the regularity
	\begin{align*}
		r(E)=\begin{cases}
			\frac{\abs{E}}{\diam(E)P(E)} &\text{ if }\abs{E}>0 \\
			0 &\text{ else}
		\end{cases}
	\end{align*}
	\end{definition}
	\begin{definition}[$\e$-isoparametric]
	We call $E\in\BV_c$ $\e$-isoparametric if for all $T\in\BV$
	\begin{align*}
		\min\brk[c]*{P(E\cap T),P(E\setminus T)}\leq \frac{P(T,\text{in }E)}{\e}\,.
	\end{align*}
	\end{definition}
	\begin{definition}[Gauge]
	We call a set thin if it is $\sigma$-finite w.r.t.\ $\cH^{n-1}$. A mapping $\delta\colon\R^n\to\R^n_{>0}$ for which $\brk[c]{\delta=0}$ is thin is called a gauge.
	\end{definition}
\end{frame}

\begin{frame}
	\begin{definition}[Partitions]
	Let $\delta$ be a gauge and $\e>0$.
	We call
	\begin{align*}
		\cP=\brk[c]*{(E_1,x_1),\dots, (E_p,x_p)}
	\end{align*}
	a partition of the set $\bigcup\cP=\bigcup_iE_i$ if $E_i\in\BV_c$ are disjoint sets and $x_i\in\R^n$. A partition is called
	\begin{itemize}
		\item dyadic if $E_i$ is a dyadic cube and $x_i\in E_i$ for all $i$
		\item $\e$-regular if $r(E_i\cup\brk[c]{x_i})>\e$ for all $i$
		\item strongly $\e$-regular if it is $\e$-regular, $E_i$ is $\e$-isoperimetric and $x_i\in\closure_*E_i$ for all $i$
		\item $\delta$-fine if $E_i\subseteq B_{\delta(x_i)}(x_i)$ for all $i$
	\end{itemize}
	\end{definition}
\end{frame}

\begin{frame}
	\begin{definition}[$R^*$-integral]
	A function $f\colon \R^n\to\R$ is called $R^*$-integrable with respect to a charge $G$ if there is a charge $F$, s.t.\ for all $\e>0$ there exists a gauge $\delta$ such that
	\begin{align*}
		\sum_{i=1}^p\abs{f(x_i)G(E_i)-F(E_i)}<\e
	\end{align*}
	for each strongly $\e$-regular $\delta$-fine partition $\brk[c]*{(E_1,x_1),\dots,(E_p,x_p)}$. We call $F$ an indefinite integral of $f$ with respect to $G$ and write
	\begin{align*}
		F=(R^*)\int f\dif G\,.
	\end{align*}
	\end{definition}
\end{frame}


\section{Uniqueness of the integral}
\begin{frame}
	\begin{proposition}
	The integral is unique.
	\end{proposition}
\end{frame}

\begin{frame}
	\begin{proposition}[Linearity of the integral]
	Let $f_1,f_2$ be $R_*$-integrable and $a\in\R$. Then $f_1+af_2$ is also integrable and
	\begin{align*}
		(R^*)\int f_1\dif G + (R^*)\int f_2\dif G=(R^*)\int f_1+\alpha f_2\dif G\,.
	\end{align*}
	\end{proposition}
\end{frame}

\section{What the integral is good for}
\begin{frame}
	\begin{proposition}[Generalisation of the Lebesgue integral on $\R^n$]
	Each Lebesgue-integrable function is also $R_*$-integrable and the integrals conincide.
	\end{proposition}
	\begin{proof}
	See \cite[Proposition 3.5]{Pfe2016}.
	\end{proof}
\end{frame}

\begin{frame}
	\begin{proposition}[Generalisation of the Henstock-Kurzweil integral on $\R$]
	A function $f\colon\R\to\R$ is Denjoy-Perron integrable on a compact $A\subseteq\R$ iff it is $R_*$-integrable and the two integrals coincide on $A$.
	\end{proposition}
	\begin{proof}
	See \cite[Proposition 3.6]{Pfe2016}.
	\end{proof}
\end{frame}

\begin{frame}
	\begin{definition}[Admissable sets]
	We call a set admissable if $\interiour_*A\subseteq A\subseteq \closure_*A$ and $\partial A$ is compact. The set of admissible $\BV$-sets is denoted by $\ABV$.
	\end{definition}
	
	\begin{proposition}[Generalisation of the Pfeffer-Integral on $\R^n$]\label{pr:GeneralisationPfeffer}
	Let $A\in\ABV$. Then each Pfeffer-integrable function is also $R_*$-integrable and the integrals coincide on $A$.
	\end{proposition}
\end{frame}

\begin{frame}
	\begin{theorem}[Divergence theorem]
	Let $A\in\ABV$, $S\subseteq A$ a thin set and $w\in C(\closure(A);\R^n)$ a continuous vector field which is pointwise Lipschitz on $A\setminus S$. Then $\diver w$ is $R_*$-integrable and
	\begin{align*}
		(R^*)\int_A\diver w\dif\cL^n=\int_{\partial_* A}w\cdot \nu_A\dif \cH^{n-1}
	\end{align*}
	where $\nu_A\colon \partial_* A\to S^{n-1}\subseteq\R^n$ is the unit normal to $A$.
	\end{theorem}
\end{frame}


\section{Summary}

\section*{Sources}

\nocite{*}

\begin{frame}
	\frametitle{Main source}
%	\bibliographystyle{plain}
%	\bibliography{bibliography}
	\printbibliography[title={Main source},keyword={main}]
\end{frame}

\begin{frame}[allowframebreaks]
	\frametitle{Other sources}
	\printbibliography[keyword={secondary}, title={Other sources}]
\end{frame}

\begin{frame}[plain]
	\begin{center}
		\Large{{Thank you for your attention.}}
	\end{center}
\end{frame}

%\frame[plain]

\end{document}
