

\input{../Preamble_Template}


%%%%% TITLE PAGE

\subject{Specialised Course in Integration Theory, VT23}
\title{Take Home Exam}
\author{Theo Koppenhöfer}
\date{\today}


\addbibresource{bibliography.bib}

%%%%% The content starts here %%%%%%%%%%%%%


\begin{document}

\maketitle

\section{Problem 1}
\subsection{a)}
\begin{claim}
The definition for a step function given in the exam and the definition given in \cite[Chapter 3.4]{CohnMT} are equivalent.
\end{claim}

\begin{proof}
Let $f$ be a step function on $[a,b]$ as given in the exam, i.e.
\begin{align}
	f=\sum_{k=1}^K c_k\one_{I_k}\label{def:f_exam}
\end{align}
for some intervals $I_k\subseteq[a,b]$. Let $\overline{I}_k=[s_k, t_k]$ and let $$a\colon\brk[c]{1,\dots,n}\to\brk[c]{a,b,s_1,\dots,s_K,t_1,\dots,t_K}$$ be a monotone bijection, i.e.\ we have for $j<k$ that also $a_j< a_k$. It then follows that
$a=a_0<a_1<\dots<a_n=b$. Furthermore we have that $f$ is constant on each interval $(a_j,a_{j+1})$ as $I_k\cap(a_j,a_{j+1})$ is either empty or the entire interval $(a_j,a_{j+1})$ and as $f$ is given by equation \ref{def:f_exam}. Thus $f$ is a step function as given in \cite[Chapter 3.4]{CohnMT}.

Let on the other hand $f$ be a step function as given in \cite[Chapter 3.4]{CohnMT}. Then there exist $a=a_0<a_1<\dots<a_n=b$ such that $f$ is constant on each interval $I_j=(a_{j-1},a_{j})$. We thus have that
\begin{align*}
	f=\sum_{j=1}^nc_j\one_{I_j}+\sum_{j=0}^nf(a_j)\one_{\brk[c]{a_j}}
\end{align*}
where $c_j=f(x_k)$ for some $x_k\in(a_{j-1},a_j)$. But this is precisely of the form \ref{def:f_exam} where we interpret $\brk[c]{a_j}=[a_j,a_j]$ to be an interval. Hence $f$ is a step function as given in the exam. Thus the claim of the equivalence of the definitions follows.
\end{proof}

\subsection{b)}
\begin{claim}
Let $\mu$ be a finite Borel-measure on the closed bounded interval $X=[a,b]$ and let $1\leq p<\infty$. Then the space of step functions on $X$ is dense in $L^p=L^p(X,\mu)$.
\end{claim}

\begin{proof}
We follow the strategy from \cite[Proposition 3.4.3]{CohnMT}. Let $f=\one_A$ be a characteristic function and let $\e>0$.
We know that $X$ is a locally compact Hausdorff space with a countable basis for its topology. Thus by \cite[Proposition 7.1.5]{CohnMT} each open subset of $X$ is $F_\sigma$. It follows with \cite[Lemma 7.2.4]{CohnMT} that there exists a closed set $F\subseteq A$ and an open set $A\subseteq U\subseteq X$ such that
\begin{align*}
	\mu(A)\leq \mu(F)+\e \qquad\text{ and }\qquad
	\mu(U)\leq \mu(A)+\e
\end{align*}
Since $X$ is compact we have that $F$ is also compact. Since $U$ is open for every $x\in U$ there exists an open interval $I_x\subseteq U$ containing $x$. Then $\brk[c]*{I_x}_{x\in U}$ is an open covering of $F$ and thus there exists a finite subcover $I_1,\dots,I_K$ of $F$. Setting $I=\bigcup_{k=1}^KI_k$ we see that $g=\one_{I}$ is a step function and $F\subseteq I\subseteq U$. Thus we have that
\begin{align*}
	\norm{f-g}_{p}^p
	&= \int_X\abs*{\one_A-\one_I}^p\dif \mu \\
	&= \int_X\one_{A\setminus I\cup I\setminus A}\dif \mu \\
	&= \mu(A\setminus I\cup I\setminus A) \\
	&\leq \mu(A\setminus I)+\mu(I\setminus A) \\
	&\leq \mu(A\setminus F)+\mu(U\setminus A) \\
	&\leq 2\e
\end{align*}
Since $\e>0$ was arbitrary it follows that the set of step functions is dense in the set of simple functions in the space of $L^p$. It follows from \cite[Proposition 3.4.2]{CohnMT} that the set of simple functions is dense in $L^p$ and as the set of step functions is dense in the set of simple functions we thus have that the set of step function is dense in $L^p$.
\end{proof}

\subsection{c)}
\begin{claim}
Let $\cC$ be as in the exam and set 
\begin{align*}
	A=\brk[c]*{f\in L^p=L^p([a,b],\mu)\colon\norm{f}_\infty\leq 1}
\end{align*}
then we have that $\closure(\cC)=A$.
\end{claim}
\begin{proof}
Let $f\in A$.
By b) there exists a sequence of step functions $\cC\ni g_k\to f$ in $L^p$ as $k\to\infty$. Set $E_k=(\abs{g_k})^{-1}([1,\infty))\subseteq [a,b]$ and note that $E_k$ is a finite union of intervals as $g_k$ is a step function.
We now define
\begin{align*}
	h_k=g_k\one_{E_k^\complement}+\frac{g_k}{\abs{g_k}}\one_{E_k}\,.
\end{align*}
By construction $h_k$ fulfills $\norm{h_k}_\infty\leq 1$. As $E_k$ is a finite union of intervals and $g_k$ and $g_k/\abs{g_k}$ are a step functions we also have that $h_k$ is a step function so $h_k\in\cC$.
Now one calculates
\begin{align*}
	\norm{f-h_k}_p^p
	= \int_{E_k^\complement}\abs*{f-h_k}^p\dif\mu + \int_{E_k}\abs*{f-h_k}^p\dif\mu \\
	= \int_{E_k^\complement}\abs*{f-g_k}^p\dif\mu + \int_{E_k}\abs*{f-\frac{g_k}{\abs{g_k}}}^p\dif\mu
\end{align*}
We now have for $\lambda$-a.e.\ $x\in E_k$ that $f(x)\in B_1=B_1(0)$ is contained in the unit ball. Since $B_1(0)$ is convex and $g_k(x)/\abs{g_k(x)}$ is the projection of $g_k(x)$ onto $B_1(0)$ it then follows that for almost every $x\in E_k$
\begin{align*}
	\abs*{f(x)-\frac{g_k(x)}{\abs{g_k(x)}}}\leq \abs*{f(x)-g_k(x)}\,.
\end{align*}
We thus have that
\begin{align*}
	\norm{f-h_k}_p^p
	&= \int_{E_k^\complement}\abs*{f-g_k}^p\dif\mu + \int_{E_k}\abs*{f-\frac{g_k}{\abs{g_k}}}^p\dif\mu \\
	&\leq \int_{E_k^\complement}\abs*{f-g_k}^p\dif\mu + \int_{E_k}\abs*{f-g_k}^p\dif\mu \\
	&= \norm{f-g_k}_p^p\xrightarrow{k\to\infty}0
\end{align*}
and hence $h_k\to f$ in $L^p$. Thus we have shown that $A\subseteq \closure(\cC)$.

Let now $f_k\to f$ in $L^p$ such that $\norm{f_k}_\infty\leq 1$. Then also $f\in L^p$. Assume now that $\norm{f}_\infty>1$. Then there exists an $\e>0$ and a set $E$ of positive measure such that $\abs{f}\geq 1+\e$ on $E$. But then
\begin{align*}
	\norm{f-f_k}_p^p
	&\geq \int_E\abs{f-f_k}^p\dif\mu  \\
	&\geq \int_E(\abs{f}-\abs{f_k})^p\dif\mu \\
	&\geq \int_E(1+\e-1)^p\dif\mu \\
	&= \int_E\e^p\dif\mu \\
	&= \mu(E)\e^p>0
\end{align*}
and hence $f_k\not\to f$ as $k\to\infty$, a contradiction. Thus we must also have that $\norm{f}_\infty\leq 1$. We have therefore shown that $A$ is closed. Since $\cC\subseteq A$ it then follows that
\begin{align*}
	\closure(\cC)\subseteq \closure(A)=A\,.
\end{align*}
\end{proof}

\subsection{d)}
\begin{claim}
$\cS$ is dense in $L^p=L^p(\R,\cB(\R),\mu)$ for any Borel measure $\mu$ that is finite on compact sets.
\end{claim}
\begin{proof}
Let $f\in L^p$ and $\e>0$. Then there exists a finite interval $K=[a,b]$ such that
\begin{align*}
	\norm{f}_{L^p\brk*{K^\complement}}^p\leq\e
\end{align*}
As $K$ is a compact interval we have that $\mu$ is a finite Borel measure on $K$ and it thus follows from b) that there exists a step function $g\in\cS$ with $\supp g\subseteq K$ such that
\begin{align*}
	\norm{f-g}_{L^p(K)}^p\leq \e\,.
\end{align*}
Putting it all together we get
\begin{align*}
	\norm{f-g}_{L^p}^p
	= \norm{\underbrace{f-g}_{\substack{=f \text{ as}\\ \supp g\subseteq K}}}_{L^p\brk*{K^\complement}}^p +\norm{f-g}_{L^p(K)}^p
	\leq 2\e
\end{align*}
and since $f\in L^p$ and $\e>0$ where arbitrary it follows that $\cS$ is dense in $L^p$.

\end{proof}

\subsection{e)}
\begin{claim}
Let $\mu$ be as in d) and let $\cS_{rc}$ be the set of right-continuous step functions that vanish at $\pm\infty$. Then $\cS_{rc}$ is dense in $L^p(\R,\cB(\R),\mu)$ for any $1\leq p<\infty$.
\end{claim}
\begin{proof}
We first show that $\cS_{rc}$ is dense in the set of step functions of the form $f=\one_{\brk[c]{a}}$ for some $a\in\R$. It follows for $\delta>0$ that
\begin{align*}
	\norm*{\one_{[a,a+\delta)}-\one_{\brk[c]{a}}}^p_p
	&= \int_{\R} \abs*{\one_{[a,a+\delta)}-\one_{\brk[c]{a}}}^p\dif \mu \\
	&= \int_{\R} \abs*{\one_{(a,a+\delta)}}^p\dif \mu \\
	&= \int_{\R} \one_{(a,a+\delta)}\dif \mu \\
	&= \mu((a,a+\delta))
\end{align*}
and further with \cite[Proposition 1.2.5]{CohnMT}
\begin{align*}
	\lim_{\delta\to0}\norm*{\one_{[a,a+\delta)}-\one_{\brk[c]{a}}}^p_p
	= \lim_{\delta\to0}\mu((a,a+\delta))
	= \mu\brk*{\bigcap_{\delta\to0}(a,a+\delta)}
	= \mu\brk*{\emptyset} =0
\end{align*}
from which it follows that $\cS_{rc}$ is dense in the set of step functions.

We now show that $\cS_{rc}$ is dense in the set of step functions of the form
\begin{align}
	\sum_{k=1}^Kc_k\one_{\brk[c]{a_k}}\,.\label{eq:dirac_step_funcs}
\end{align}
For this let $\e>0$ and $g_k\in\cS_{rc}$ be such that $\norm*{g_k-\one_{\brk[c]{a_k}}}_p<\e_k$ where $\e_k=\e/(K\cdot \max_k\abs{c_k})$. Then it follows that also $\sum_{k=1}^Kc_kg_k\in\cS_{rc}$ and
\begin{align*}
	\norm*{\sum_{k=1}^Kc_k\one_{\brk[c]{a_k}}-\sum_{k=1}^Kc_kg_k}_p
	&\leq \sum_{k=1}^K\norm*{c_k\one_{\brk[c]{a_k}}-c_kg_k}_p \\
	&= \sum_{k=1}^K\abs{c_k}\norm*{\one_{\brk[c]{a_k}}-g_k}_p \\
	&\leq \sum_{k=1}^K\max_k\abs{c_k}\e_k \\
	&= \sum_{k=1}^K\frac{\e}{K}=\e\,.
\end{align*}
As $\e>0$ was arbitrary it follows that $\cS_{rc}$ is dense in the set of functions of the form \eqref{eq:dirac_step_funcs}.

Let now $f\in L^p$ and $\e>0$. It follows from d) that there exists $g_1\in \cS$ such that
\begin{align*}
	\norm{f-g_1}_{p}\leq \e
\end{align*}
Since $g_1\in \cS$ there exist finite intervals $I_1,\dots,I_K$ such that
\begin{align*}
	g_1=\sum_{k=1}^Kc_k\one_{I_k}
\end{align*}
Let $\overline{I_k}=[a_k,b_k]$. Then we can write 
\begin{align*}
	g_1=\sum_{k=1}^Kc_k\one_{[a_k,b_k)}+\sum_{k=1}^Kd_k\one_{\brk[c]{a_k}}+\sum_{k=1}^Ke_k\one_{\brk[c]{b_k}}
\end{align*}
for some $d_k,e_k\in\R$. Since $\cS_{rc}$ is dense for step functions of the form \eqref{eq:dirac_step_funcs} there exists a function $g_2\in\cS_{rc}$ such that
\begin{align*}
	\norm*{\sum_{k=1}^Kd_k\one_{\brk[c]{a_k}}+\sum_{k=1}^Ke_k\one_{\brk[c]{b_k}}-g_2}_p\leq \e
\end{align*}
We now set
\begin{align*}
	g= g_1 -\sum_{k=1}^Kd_k\one_{\brk[c]{a_k}}-\sum_{k=1}^Ke_k\one_{\brk[c]{b_k}}+g_2= \sum_{k=1}^Kc_k\one_{[a_k,b_k)}+g_2\in\cS_{rc}\,.
\end{align*}
Taking everything together we obtain that
\begin{align*}
	\norm*{g-f}_p
	&=\norm*{g_1 -\sum_{k=1}^Kd_k\one_{\brk[c]{a_k}}-\sum_{k=1}^Ke_k\one_{\brk[c]{b_k}}+g_2-f}_p \\
	&\leq \norm*{g_1-f}_p+\norm*{\sum_{k=1}^Kd_k\one_{\brk[c]{a_k}}-\sum_{k=1}^Ke_k\one_{\brk[c]{b_k}}+g_2}_p \\
	&\leq 2\e
\end{align*}
and since $\e>0$ was arbitrary the claim follows.
\end{proof}

\subsection{f)}
\begin{claim}
For $s\in \cS_{rc}$ there exists an increasing sequence $\brk[c]{t_k}_{k=0}^K$ and numbers  $\brk[c]{c_k}_{k=1}^K$ such that
\begin{align*}
	s = \sum_{k=1}^Kc_k\one_{[t_{k-1},t_k)}\,.
\end{align*}
\end{claim}
\begin{proof}
Since $s$ is a step function which vanishes on $\pm\infty$ there exists a finite interval $[a,b]$ such that $\supp s\subseteq(a,b)$. Then $s$ is also a step function on $[a,b]$ and by a) there exists an increasing sequence $\brk[c]{t_k}_{k=0}^K$ such that $a=t_0<t_1<\dots<t_K=b$ and such that $s$ is constant on each interval $(t_{k-1},t_k)$. Hence $s$ is of the form
\begin{align}
	s = \sum_{k=1}^Kc_k\one_{(t_{k-1},t_k)}+\sum_{k=0}^Ks(t_k)\one_{\brk[c]{t_k}}\,.\label{eq:s_shape}
\end{align}
for a sequence of numbers $\brk[c]{c_k}_{k=1}^K$. Now since $s$ is right-continuous we have for $k=0,\dots,K-1$ that
\begin{align*}
	s(t_k)=\lim_{\substack{t\to t_k\\t>t_k}}s(t)=c_k
\end{align*}
Together with the fact that $0=s(b)=s(t_K)$ we obtain from equation \ref{eq:s_shape} that
\begin{align*}
	s = \sum_{k=1}^Kc_k\one_{(t_{k-1},t_k)}+\sum_{k=0}^ {K-1}c_k\one_{\brk[c]{t_k}}
	= \sum_{k=1}^Kc_k\one_{[t_{k-1},t_k)}
\end{align*}
as claimed.
\end{proof}

\section{Problem 2}

\subsection{a)}
\begin{claim}
We have for every finite signed measure $\mu$ that $V_{F_\mu}\leq \norm{\mu}$.
\end{claim}
\begin{proof}
We have in the notation of \cite[chapter 4.4]{CohnMT} that
\begin{align*}
	V_{F_\mu}&=\sup\brk[c]*{\sum_{j=1}^n\abs*{F_\mu(t_j)-F_\mu(t_{j-1})}\colon \brk{t_j}_{j=0}^n\in\cS} \\
	&= \sup\brk[c]*{\sum_{j=1}^n\abs*{\mu\brk*{(t_{j-1},t_j]}}\colon \brk{t_j}_{j=0}^n\in\cS} \\
	&\leq \sup\brk[c]*{\sum_{j=1}^n\abs*{\mu\brk*{B_j}}\colon (B_j)_{j=1}^n\text{ are measurable and pairwise disjoint}} \\
	&=\norm{\mu}\,.
\end{align*}
\end{proof}

\subsection{b)}


\subsection{c)}
\begin{claim}
The variation $V_F=\norm{F}_{\cV}$ is a norm that makes the space $\cV$ of all right-continuous functions of finite variation that vanish at $-\infty$ into a Banach space. Furthermore the map $\Phi\colon M_r\to\cV,\mu\mapsto F_\mu$ is a linear isometry between the space of all finite signed Borel measures $M_r=M_r(\R,\cB(\R))$ and the space $(\cV,\norm{\cdot}_\cV)$.
\end{claim}
\begin{proof}
We know from \cite[Proposition 4.4.3]{CohnMT} that $\mu$ is bijective. Let $f,g\in\cV$ and $\alpha\in\R$ then there exist $\mu,\nu\in M_r$ such that $F_\mu=f$ and $F_\nu=g$. It follows from a) and b) that 
\begin{align}
	\norm{\Phi(\mu)}_\cV=V_{F_\mu}=\norm{\mu}\,. \label{eq:V_isometry}
\end{align}
Since $\Phi$ is linear we have
\begin{align}
	\begin{aligned}
	\norm{ f+g}_\cV
	&= \norm{ \Phi(\mu)+\Phi(\nu)}_\cV \\
	&= \norm{ \Phi(\mu+\nu)}_\cV \\
	&= \norm{\mu+\nu}  \\
	&\leq \norm{\mu}+\norm{\nu}  \\
	&= \norm{\Phi(\mu)}_\cV+\norm{\Phi(\nu)}_\cV \\
	&= \norm{f}_\cV+\norm{g}_\cV
	\end{aligned}
	\label{eq:V_linear}
\end{align}
and
\begin{align}
	\begin{aligned}
	\norm{\alpha f}_\cV
	&= \norm{\alpha \Phi(\mu)}_\cV \\
	&= \norm{ \Phi(\alpha\mu)}_\cV \\
	&= \norm{\alpha\mu}  \\
	&= \abs{\alpha}\norm{\mu}  \\
	&= \abs{\alpha}\norm{\Phi(\mu)}_\cV \\
	&= \abs{\alpha}\norm{f}_\cV
	\end{aligned}
	\label{eq:V_homogeneous}
\end{align}
so $\norm{\cdot}_\cV$ fulfills the triangle inquality and homogeneity. Since $\alpha f+g$ is also right continuous and vanishes at $-\infty$ it follows from equations \ref{eq:V_linear} and \ref{eq:V_homogeneous} that $\alpha f+g\in\cV$. Thus $\cV$ is a linear space. We also have that
\begin{align*}
	0=\norm{f}_\cV=\norm{\Phi(\mu)}_\cV=\norm{\mu}
\end{align*}
implies that $\mu=0$ and hence $f=0$. Thus $\norm{\cdot}_\cV$ provides $\cV$ with a norm. 
It now follows from equation \ref{eq:V_isometry} that $\Phi$ defines an isometric isomorphism. Let now $f_k=\Phi(\mu_k)$ be a Cauchy sequence in $\cV$. As $\Phi$ is an isometry we have that $\mu_k$ is Cauchy in $M_r$. From \cite[chapter 7.3]{CohnMT} we know that $M_r$ is a Banach space and thus $\mu_k\to \mu$ in $M_r$. But then we also have that $f_k\to f=\Phi(\mu)\in\cV$ by continuity of $\Phi$. Thus $\cV$ is complete and thus a Banach space.

\end{proof}
\section{Problem 3}
\subsection{a)}

\begin{claim}
Let $F_k$ be a sequence of differentiable functions such that $F=\sum_{k\in\N}F_k$ is pointwise absolutely convergent on $[a,b]$. If
\begin{align*}
	\sum_{k\in\N}\norm{F'_k}_\infty<\infty
\end{align*}
then $F'=\sum_{k\in\N}F'_k$.
\end{claim}
\begin{proof}
Let $x_n\to x$ in $[a,b]$ as $n\to\infty$. Then we have
\begin{align*}
	\frac{F(x)-F(x_n)}{x-x_n}
	= \frac{\sum_kF_k(x)-\sum_kF_k(x_n)}{x-x_n}
	= \sum_k\frac{F_k(x)-F_k(x_n)}{x-x_n}
\end{align*}
where the reordering of the sums is permitted as the sums are pointwise absolutely convergent. Now we have by standard estimates  for differentiable functions that
\begin{align*}
	\frac{\abs*{F_k(x)-F_k(x_n)}}{\abs*{x-x_n}}
	\leq \frac{\norm{F'_k}_\infty\abs*{x-x_n}}{\abs*{x-x_n}}
	= \norm{F'_k}_\infty
\end{align*}
Hence the sequence
\begin{align*}
	\brk*{\frac{F_k(x)-F_k(x_n)}{x-x_n}}_{k\in\N}
\end{align*}
is bounded from above by the summable sequence $\brk*{\norm{F'_k}_\infty}_{k\in\N}$.
We also have for every $k\in\N$ that
\begin{align*}
	\lim_{n\to\infty}\frac{F_k(x)-F_k(x_n)}{x-x_n}=F'_k(x)\,.
\end{align*}
It then follows from Lebesgue's dominated convergence theorem that
\begin{align*}
	\lim_{n\to\infty}\frac{F(x)-F(x_n)}{x-x_n}
	= \lim_{n\to\infty}\sum_k\frac{F_k(x)-F_k(x_n)}{x-x_n}
	= \sum_kF'_k(x)\,.
\end{align*}
As $(x_n)_{n\in\N}$ was an arbitrary sequence converging to $x$ in $[a,b]$ the claim follows.
\end{proof}

\subsection{b)}
\begin{claim}
Let $\brk[c]{q_k}_{k\in\N}$ be a dense set in $(0,1)$ and $I_k=[q_k,q_k+2^{-k}]$. Then $\lambda$-a.e.\ point in $(0,1)$ appears in at most finitely many $I_k$.
\end{claim}
\begin{proof}
We have with the monotonicity and $\sigma$-additivity of $\lambda$ that
\begin{align*}
	\lambda\brk*{\liminf_{k\in\N} I_k}
	= \lambda\brk*{\bigcap_{k}\bigcup_{n> k}I_n}
	\leq \lambda\brk*{\bigcup_{n> k}I_n}
	\leq\sum_{n> k} \lambda\brk*{I_n}
	=\sum_{n> k} 2^{-n}
	=2^{-k}\xrightarrow{k\to\infty}0
\end{align*}
So $\lambda\brk{\liminf_k I_k}=0$ and thus almost every point appears only finitely often in the intervals $I_k$.
\end{proof}

\subsection{c)}
\begin{claim}
Let
\begin{align*}
	f(x)=x^2\sin\brk*{\frac{1}{x^2}}\one_{x>0}
\end{align*}
then $f$ is differentiable on $\R$ and we have for $s<0$ and $s<t$ that
\begin{align*}
	\frac{\abs{f(t)-f(s)}}{t-s}\leq t-s\,.
\end{align*}
\end{claim}
\begin{proof}
If $s<t\leq0$ then we have that
\begin{align}
	\frac{\abs{f(t)-f(s)}}{t-s}
	\frac{\abs{0-0}}{t-s}
	= 0\leq t-s\,. \label{eq:est_t_negative}
\end{align}
If $s\leq 0<t$ then we have
\begin{align}
	\frac{\abs{f(t)-f(s)}}{t-s}
	= \frac{\abs{t^2\sin (1/t^2)}}{t-s}
	\leq \frac{t^2}{t}=t\leq t-s\label{eq:est_t_positive}
\end{align}
and the second part of the claim follows.

For differentiability we note that $f=0$ is differentiable on $\R_{<0}$ and $f=x^2\sin\brk*{1/x^2}$ is differentiable on $\R_{>0}$. Thus the differentiability of $f$ at $0$ remains to be shown. For this it follows by setting $t=0$ in equation \ref{eq:est_t_negative} and $s=0$ in equation \ref{eq:est_t_positive} that for $x\in\R\setminus\brk[c]{0}$
\begin{align*}
	\frac{\abs{f(0)-f(x)}}{\abs{0-x}}\leq \abs{x}\xrightarrow{x\to0}0
\end{align*}
and thus $f$ is also differentiable in $0$.
\end{proof}

\subsection{d)}
\begin{claim}
Define $F$ by
\begin{align}
	F=\sum_{k\in\N}F_k\label{def:F}
\end{align}
where
\begin{align*}
	F_k= c_kf(\cdot-q_k)
\end{align*}
for a suitable sequence $(c_k)_k$. Then $F$ is $\lambda$-a.e. right-differentiable on $(0,1)$. 
\end{claim}
\begin{proof}
We follow the hint.
We know that $f$ is continuously differentiable on $\R_{>0}$ and thus we can choose $c_k$ such that
\begin{align*}
	C=\sum_{k\in\N}c_k\norm{f'}_{C^0([2^{-k},1])}<\infty
\end{align*}
e.g.\ by setting $0<c_k\leq 1/(\norm{f'}_{C^0([2^{-k},1])}\cdot k^2)$. Additionally we choose $c_k$ such that
\begin{align*}
	\tiC=\sum_{k\in\N}c_k<\infty
\end{align*}
e.g.\ by additionally requiring $0<c_k\leq 1/k^2$.
It then follows from $\abs{f}\leq 1$ on $[-1,1]$ that
\begin{align*}
	\sum_{k\in\N}\abs{F_k}
	= \sum_{k\in\N}c_k\abs{f(\cdot-q_k)}
	\leq \sum_{k\in\N}c_k<\infty\,.
\end{align*}
So $\sum_kF_k$ is absolutely convergent and $F$ is in particular well-defined. By part b) there exists a measurable set $S\subseteq(0,1)$ such that $\lambda(S)=1$ and such that every $x\in S$ is contained in at most finitely many of the intervals $[q_k,q_k+2^{-k}]$. Thus for $K$ large enough we have for $s\in S$ that
\begin{align*}
	\sum_{\substack{q_k< s\\k\geq K}}\norm{F'_k}_{C^0([s,1])}
	&= \sum_{\substack{q_k< s\\k\geq K}}c_k\norm{f'(\cdot-q_k)}_{C^0([s,1])} \\
	&\leq \sum_{\substack{q_k< s\\k\geq K}}c_k\norm{f'}_{C^0([s-q_k,1])} \\
	&\leq \sum_{\substack{q_k< s\\k\geq K}}c_k\norm{f'}_{C^0([2^{-k},1])} \\
	&\leq C\,.
\end{align*}
As $\sum_kF_k$ is an absolutely convergent series it follows with a) that
\begin{align*}
	G=\sum_{\substack{q_k< s\\k\geq K}} F_k
\end{align*}
is differentiable on $[s,1]$.
As we have $\supp F_k\subseteq [q_k,1]$ it follows for $s< t$ that
\begin{align*}
	F(s)-F(t)
	&=\sum_kF_k(s)-\sum_kF_k(t) \\
	&=\sum_{q_k< s}F_k(s)-\brk*{\sum_{q_k<s}+\sum_{s\leq q_k <t}}F_k(t) \\
	&=\sum_{q_k< s}\brk*{F_k(s)-F_k(t)}-\sum_{s\leq q_k <t}F_k(t) \\
	&=\brk*{\sum_{\substack{q_k< s\\k< K}}+\sum_{\substack{q_k< s\\k\geq K}}}\brk*{F_k(s)-F_k(t)}+\sum_{s\leq q_k <t}F_k(t) \\
	&=\sum_{\substack{q_k< s\\k< K}}\brk*{F_k(s)-F_k(t)}+G(s)-G(t)+\sum_{s\leq q_k <t}F_k(t) \\
\end{align*}
Division by $(s-t)$ yields
\begin{align*}
	\frac{F(s)-F(t)}{s-t}
	=\underbrace{\sum_{\substack{q_k< s\\k< K}}\frac{F_k(s)-F_k(t)}{s-t}}_{=\text{(I)}}+\underbrace{\frac{G(s)-G(t)}{s-t}}_{=\text{(II)}}+\underbrace{\sum_{s\leq q_k <t}\frac{F_k(t)}{s-t}}_{=\text{(III)}}
\end{align*}
To show that $F$ is right-differentiable in $S\subseteq[0,1]$ it suffices to show that the terms (I), (II) and (III) converge as $t\to s$ in the expression above. 
We note that the first term converges as $t\to s$ since by c) the $F_k$ are differentiable and since the sum is finite.
The second term converges because $G$ is differentiable.
For the third term it follows from c) that
\begin{align*}
	\abs{\text{(III)}}&\leq\sum_{s\leq q_k <t}\frac{\abs{F_k(t)}}{\abs{s-t}} \\
	&=\sum_{s\leq q_k <t}\frac{\abs*{F_k(t)-F_k(s)}}{\abs{s-t}} \\
	&=\sum_{s\leq q_k <t}c_k\frac{\abs*{f(t-q_k)-f(s-q_k)}}{\abs{t-q_k-(s-q_k)}}	 \\
	&\leq\sum_{s\leq q_k <t}c_k\abs{t-q_k-(s-q_k)} \\
	&=\sum_{s\leq q_k <t}c_k\abs*{t-s} \\
	&\leq\tiC \abs*{t-s}\xrightarrow{t\to s}0
\end{align*}
and thus the claim follows.
\end{proof}

\section{Problem 4}
\subsection{a)}
\begin{claim}
The functions $f\in C(X)$ which have a continuous extension to the one-point compactification $X^*$ of $X$ are precisely those for which
\begin{align}
	\lim_{x\to k}f(x)=\lim_{\abs{x}\to\infty}f(x)=c\in\R\label{eq:1pt_cptcation_cond}\,.
\end{align}
for all $k\in\Z$.
\end{claim}
\begin{proof}
Let $f\in C(X)$ be such that $f$ has an extension $f^*$ to $X^*$. Then we have for all open neighbourhoods $U\subseteq\R$ of $c$ that $$\brk*{f^*}^{-1}(U)=f^{-1}(U)\cup\brk[c]{\infty}=V\cup \brk[c]{\infty}$$ is open in $X$. As $\infty$ is contained in this open set it follows that $K^\complement\subseteq V$ for some compact $K\subseteq X$. As $K\subseteq X$ and $\brk[c]{k}\subseteq \R\setminus X$ for every $k\in\Z$ are compact in $\R$ and as $\R$ is regular there exists around every $k\in\Z$ an open neighbourhood $V_k\subseteq \R$ such that $V_k\cap K=\emptyset$. Thus we have $V_k\cap X\subseteq V= f^{-1}(U)$. Since $U$ was an arbitrary open neighbourhood of $c$ this means that $\lim_{x\to k}f(x)=c$ for all $k\in\Z$. Now define open sets
$$V_\infty=(\max K, \infty)\qquad\text{ and }\qquad V_{-\infty}=(-\infty,\min K)\,.$$
It then follows from $V_\infty\cap K=\emptyset=V_{-\infty}\cap K$ that 
$$V_\infty\cap X\subseteq V=f^{-1}(U) \qquad\text{ and }\qquad V_{-\infty}\cap X\subseteq V=f^{-1}(U)\,.$$
As $U$ was an arbitrary open neighbourhood of $c$ this means that $\lim_{\abs{x}\to\infty}f(x)=c$. Thus $f$ has to fulfill condition \ref{eq:1pt_cptcation_cond}.


Let now $f\in C(X)$ be such that $f$ fulfills condition \ref{eq:1pt_cptcation_cond}. Set $f^*=f$ on $X$ and $f^*(\infty) = c$. 
%Let now $x_k\to x$ in $X^*$ as $k\to\infty$. 
%If $x=\infty$ then we have that $f^*(x_k)\to c=f^*(x)$ by assumption. 
Let $U\subseteq \K$ be open. Now if $c\notin U$ then $\brk*{f^*}^{-1}(U)=f^{-1}(U)$ is open by continuity of $f$. Else $\brk*{f^*}^{-1}(U)=f^{-1}(U)\cup\brk[c]{\infty}$. It follows from condition \ref{eq:1pt_cptcation_cond} that there exist for all $k\in\Z$ open neighbourhoods $V_k\subseteq X$ of $k$ such that $f(V_k)\subseteq U$. By the same reason there exist open intervals $V_\infty=(k_+, \infty)$ and $V_{-\infty}=(-\infty,k_{-})$ such that
$$f(V_\infty\cap X)\subseteq U \qquad\text{ and }\qquad f(V_{-\infty}\cap X)\subseteq U\,.$$
Now define
$$K=X\setminus \brk*{\brk*{\bigcup_{k\in\Z}V_k}\cup V_\infty\cup V_{-\infty}}\,.$$
Then $K$ is bounded and closed in $\R$ by construction and thus compact in $\R$. As $K\cap\Z=\emptyset$ it is also compact in $X$ and we have that 
$$K^\complement=\brk*{\brk*{\bigcup_{k\in\Z}V_k}\cup V_\infty\cup V_{-\infty}}\subseteq f^{-1}(U)$$
and hence 
$$\brk*{f^*}^{-1}(U)=f^{-1}(U)\cup\brk[c]{\infty}=f^{-1}(U)\cup \brk*{K^\complement\cup\brk[c]{\infty}}$$
is open. Thus $f^*$ is continuous in $X^*$.
\end{proof}

\subsection{b)}

\begin{claim}
The set $C_0(X)^*$ is a closed linear subspace of $C_0(\R)$.
\end{claim}
\begin{proof}
Let $f\in C_0(X)$. Set $f^*=f$ on $X$ and $f^*=0$ on $\Z$. We claim that $f^*\in C_0(\R)$. 

Indeed let $U\subseteq \R$ be open. If $0\notin U$ then $\brk*{f^*}^{-1}(U)=f^{-1}(U)$ which is open in $X$ and hence also open in $\R$. If $0\in U$ then $\brk*{f^*}^{-1}(U)=f^{-1}(U)\cup \Z$. As $f\in C_0(X)$ there exists a compact $K\subseteq X$ such that $f\brk*{X\setminus K}\subseteq U$. It then follows that
$$\brk*{f^*}^{-1}(U)=f^{-1}(U)\cup \Z=f^{-1}(U)\cup X\setminus K\cup\Z=f^{-1}(U)\cup\R\setminus K$$
is also open in $\R$. Hence $f^*\in C(\R)$. It also follows that $f^*(\R\setminus K)=\brk[c]{0}\cup f(\R\setminus K)\subseteq U$. Since $U$ was an arbitrary neighbourhood of $0$ we in fact have $f^*\in C_0(\R)$.

It remains to show that $C_0(X)$ embedded into $C_0(\R)$ in this way forms a closed linear subspace. For this let $f^*,g^*\in C_0(\R)$ such that $f^*=0$ on $\Z$. Then we have for $\alpha\in\R$ that $\alpha f^*+g^*\in C_0(\R)$ and $\alpha f^*+g^*=0$ on $\Z$ so $C_0(X)$ is indeed a linear subspace. Furthermore let $f_k^*\in C_0(\R)$ be a Cauchy sequence such that $f_k^*=0$ on $\Z$ for every $k\in\Z$. Then $f_k^*$ converges to some $f^*\in C_0(\R)$. In particular $f_k^*$ converges pointwise on $\Z$ to $0$ and thus $f^*=0$ on $\Z$. Thus $C_0(X)$ is a closed subspace of $C_0(\R)$.
\end{proof}

\begin{claim}
$C_0(X)^*$ can be identified with the space $S$ of finite signed Borel measures on $\R$ $\mu\in M_r(\R)=M_r(\R,\cB(R),\R)$ such that $\abs{\mu}(\Z)=0$.
\end{claim}

\begin{proof}
We know from \cite[Theorem 7.3.6]{CohnMT} that $C_0(X)^*$ can be identified with the space of finite signed Borel measures on $X$ denoted by $M_r(X)$. We now claim that $M_r(X)$ can be identified with the space $S$. Indeed if $\hmu\in M_r(X)$ then $\mu=\hmu(\cdot\cap X)\in M_r(\R)$ and $\abs{\mu}(\Z)=0$ as for all $Z\subseteq\Z$ we have that
\begin{align*}
	\mu(Z)=\hmu(Z\cap X) =\hmu(\emptyset)=0\,.
\end{align*}
On the other hand if $\mu\in M_r(\R)$ such that $\abs{\mu}(\Z)=0$ then $\mu\in M_r(X)$. Additionally the mapping $\hmu\mapsto\mu$ is injective and thus an embedding.

It remains to be shown that $S$ is a closed linear space. For this let $\alpha\in\R$ and $\mu, \nu\in S$ then
\begin{align*}
	\abs{\alpha\mu+\nu}(\Z)\leq \alpha\abs{\mu}(\Z)+\abs{\nu}(\Z)=0
\end{align*}
and thus $\alpha\mu+\nu\in S$ and $S$ is a linear space. Let now $\mu_k\in S$ be a Cauchy sequence in $M_r(\R)$. By completeness $\mu_k$ converge to some $\mu\in M_r(\R)$ as $k\to\infty$ (c.f.\ \cite[Chapter 7.3]{CohnMT}). It then follows that
\begin{align*}
	\abs{\mu}(\Z)=\abs{\mu-\mu_k}(\Z)\leq \norm{\mu-\mu_k}\xrightarrow{k\to\infty}0
\end{align*}
so also $\mu\in S$. Hence $S$ is closed.
\end{proof}

\subsection{c)}
\begin{claim}
The set $C_0(X)^\perp$ can be identified (via the identification in b)) with the subspace of measures $\mu\in M_r(\R)$ whose support is contained in $\Z$.
\end{claim}
\begin{proof}
Let $\mu$ be a measure in $C_0(X)^\perp$. Then we have for all $f\in C^0(X)$ that $\brk[a]{\hmu,f}=0$ where we write $\brk[a]{\hmu,f}=\int_Xf\dif\hmu$. This implies in particular that for all measurable $A\subseteq X$ we have that
\begin{align*}
	0=\brk[a]{\hmu,\one_A}=\hmu(A)=\mu(A)\,.
\end{align*}
Hence $\abs{\mu}(X)=0$ and thus $\supp\mu\subseteq\overline{\R\setminus X}=\Z$.

On the other hand let $\mu\in M_r(\R)$ be such that $\supp\mu\subseteq\Z=\R\setminus X$ then we have for all measurable $A\subseteq X$ that $\hmu(A)=\mu(A)=0$ so $\hmu=0$ on $X$ and thus for all $f\in C_0(X)$ we have that $\brk[a]{\hmu,f}=0$. Hence $\mu$ is a measure in $C_0(X)^\perp$.

It remains to show that $C_0(X)^\perp$ forms a linear subspace. For this let $\alpha\in\R$ and $\mu,\nu\in M_r(\R)$ such that $\supp\mu\subseteq\Z$ and $\supp\nu\subseteq\Z$. Then also $\alpha\mu+\nu\in M_r(\Z)$ and we have that $\supp(\alpha\mu+\nu)\subseteq\Z$ and the claim follows.
\end{proof}

\subsection{d)}
\begin{claim}
The measures in $C_0(X)^\perp$ are singular with respect to $\lambda$.
\end{claim}
\begin{proof}
Let $\mu\in C_0(X)^\perp$ then by c) we have $\supp\mu\subseteq\Z$. Since this means that $\abs{\mu}(\R\setminus\Z)=0$ we have that $\abs{\mu}$ is concentrated on $\Z$. Now as $\Z$ is a $\lambda$-nullset we have that $\lambda$ is concentrated on $\R\setminus\Z$ so $\mu$ and $\lambda$ are singular.
\end{proof}

\begin{claim}
The measures in $C_0(X)^\perp$ are absolutely continuous with respect to the counting measure $\nu$ on $\Z$.
\end{claim}
\begin{proof}
Let $\mu\in C_0(X)^\perp$ then and let $f$ be measurable with respect to the counting measure such that $f(k)=\mu(\brk[c]{k})$ for all $k\in\Z$. E.g.\ we could take 
$$f=\mu(\brk[c]{k})\one_{[k-1/2,k+1/2)}\,.$$
By c) we have that $\supp \mu\subseteq\Z$ and it follows for all measurable $A\subseteq X$ using the $\sigma$-additivity of $\mu$ that
\begin{align*}
	\mu(A)=\mu\brk*{A\cap \Z}
	=\mu\brk*{\bigcup_{k\in A\cap \Z}\brk[c]{k}}
	=\sum_{k\in A\cap \Z}\mu(\brk[c]{k})
	=\sum_{k\in A\cap \Z}f(k)
	=\int_Af\dif \nu\,.
\end{align*}
Thus $\mu$ is absolutely continuous with respect to $\nu$.
\end{proof}

\subsection{e)}
\begin{claim}
If we combine the answers from b) and c) we get that every measure $\mu\in M_r(\R)$ can be decomposed into $\mu=\mu_1+\mu_2$ where $\mu_1\in C_0(X)^*$ and $\mu_2\in C_0(X)^\perp$ are mutually singular measures. We can write this as $M_r(\R)=C_0(X)^*\oplus C_0(X)^\perp$.
\end{claim}
\begin{proof}
For $\mu\in M_r(\R)$ set $\mu_1=\mu(\cdot\cap X)$ and $\mu_2=\mu(\cdot\cap \Z)$. We have for all measurable sets $A\subseteq\R$ that
\begin{align*}
	\mu(A) = \mu(A\cap X)+\mu(A\cap \Z)=\mu_1(A)+\mu_2(A)\,.
\end{align*}
Also $\mu_1$ is concentrated on $X$ and $\mu_2$ is concantrated on $\Z$ so $\mu_1$ and $\mu_2$ are mutually singular. As $\mu_2$ is concentrated on $\Z$ we have that $\supp\mu_2\subseteq\Z$. It then follows from c) that $\mu_2\in C_0(X)^\perp$. It also follows from $\mu_1(Z)=0$ for all $Z\subseteq \Z$ that $\abs{\mu_1}(\Z)=0$. Thus we have by b) that $\mu_1\in C_0(X)^*$.
\end{proof}



%\section*{Bibliography}
\nocite{*}
\printbibliography

\end{document}